%% Generated by Sphinx.
\def\sphinxdocclass{jsbook}
\documentclass[letterpaper,10pt,dvipdfmx,openany]{sphinxmanual}
\ifdefined\pdfpxdimen
   \let\sphinxpxdimen\pdfpxdimen\else\newdimen\sphinxpxdimen
\fi \sphinxpxdimen=.75bp\relax

\PassOptionsToPackage{warn}{textcomp}


\usepackage{cmap}
\usepackage[T1]{fontenc}
\usepackage{amsmath,amssymb,amstext}

\usepackage{times}

\usepackage[,numfigreset=1,mathnumfig]{sphinx}

\fvset{fontsize=\small}
\usepackage[dvipdfm]{geometry}

% Include hyperref last.
\usepackage{hyperref}
% Fix anchor placement for figures with captions.
\usepackage{hypcap}% it must be loaded after hyperref.
% Set up styles of URL: it should be placed after hyperref.
\urlstyle{same}
\renewcommand{\contentsname}{目次:}

\renewcommand{\figurename}{図 }
\makeatletter
\def\fnum@figure{\figurename\thefigure{}}
\makeatother
\renewcommand{\tablename}{表 }
\makeatletter
\def\fnum@table{\tablename\thetable{}}
\makeatother
\renewcommand{\literalblockname}{リスト}

\renewcommand{\literalblockcontinuedname}{前のページからの続き}
\renewcommand{\literalblockcontinuesname}{次のページに続く}
\renewcommand{\sphinxnonalphabeticalgroupname}{Non-alphabetical}
\renewcommand{\sphinxsymbolsname}{記号}
\renewcommand{\sphinxnumbersname}{Numbers}

\def\pageautorefname{ページ}

\setcounter{tocdepth}{2}


\usepackage{enumitem}
\setlistdepth{2048}
\setlength{\baselineskip}{10pt}
\renewcommand{\baselinestretch}{0.65}


\title{MTC-to-DenD デザインディシジョンドキュメント}
\date{2020年12月30日}
\release{0.0.0}
\author{saha209}
\newcommand{\sphinxlogo}{\vbox{}}
\renewcommand{\releasename}{リリース}
\makeindex
\begin{document}

\pagestyle{empty}
\sphinxmaketitle
\pagestyle{plain}
\sphinxtableofcontents
\pagestyle{normal}
\phantomsection\label{\detokenize{index::doc}}



\chapter{MTC-to-DenD デザインディシジョン}
\label{\detokenize{doc/designDecision:mtc-to-dend}}\label{\detokenize{doc/designDecision::doc}}

\section{はじめに}
\label{\detokenize{doc/designDecision:id1}}
MTC-to-DenD のプロジェクト実装する上で、どうしてそのコードを書いたかを記す。


\section{コールグラフ}
\label{\detokenize{doc/designDecision:id2}}
\sphinxincludegraphics{mermaid-6e91136ba5ed9e337339a1b4969d934fdb3ca3db.pdf}


\section{まとめ}
\label{\detokenize{doc/designDecision:id3}}
MTC-to-DenD のプロジェクト実装する上で、どうしてそのコードを書いたかを記した。


\subsection{残件}
\label{\detokenize{doc/designDecision:id4}}\begin{itemize}
\item {} 
ほかのMTCに対応する
\begin{quote}

ベンダIDとデバイスIDの変更に対応してみたが、マスターコントローラーがどの値を返すかが分からないと対応ができない。
\end{quote}

\item {} 
任意キーへのマッピング

\end{itemize}


\chapter{Indices and tables}
\label{\detokenize{index:indices-and-tables}}\begin{itemize}
\item {} 
\DUrole{xref,std,std-ref}{genindex}

\item {} 
\DUrole{xref,std,std-ref}{modindex}

\item {} 
\DUrole{xref,std,std-ref}{search}

\end{itemize}



\renewcommand{\indexname}{索引}
\printindex
\end{document}